%===============================================================================
% LaTeX sjabloon voor de bachelorproef toegepaste informatica aan HOGENT
% Meer info op https://github.com/HoGentTIN/latex-hogent-report
%===============================================================================

\documentclass[dutch,dit,thesis]{hogentreport}

% TODO:
% - If necessary, replace the option `dit`' with your own department!
%   Valid entries are dbo, dbt, dgz, dit, dlo, dog, dsa, soa
% - If you write your thesis in English (remark: only possible after getting
%   explicit approval!), remove the option "dutch," or replace with "english".

\usepackage{lipsum} % For blind text, can be removed after adding actual content

%% Pictures to include in the text can be put in the graphics/ folder
\graphicspath{{graphics/}}

%% For source code highlighting, requires pygments to be installed
%% Compile with the -shell-escape flag!
\usepackage[section]{minted}
%% If you compile with the make_thesis.{bat,sh} script, use the following
%% import instead:
%% \usepackage[section,outputdir=../output]{minted}
\usemintedstyle{solarized-light}
\definecolor{bg}{RGB}{253,246,227} %% Set the background color of the codeframe

%% Change this line to edit the line numbering style:
\renewcommand{\theFancyVerbLine}{\ttfamily\scriptsize\arabic{FancyVerbLine}}

%% Macro definition to load external java source files with \javacode{filename}:
\newmintedfile[javacode]{java}{
    bgcolor=bg,
    fontfamily=tt,
    linenos=true,
    numberblanklines=true,
    numbersep=5pt,
    gobble=0,
    framesep=2mm,
    funcnamehighlighting=true,
    tabsize=4,
    obeytabs=false,
    breaklines=true,
    mathescape=false
    samepage=false,
    showspaces=false,
    showtabs =false,
    texcl=false,
}

% Other packages not already included can be imported here

%%---------- Document metadata -------------------------------------------------
% TODO: Replace this with your own information
\author{Jelle Delporte}
\supervisor{ir. drs. Lieven JS Smits}
\cosupervisor{dr. ir. Gilles Vandewiele}
\title[]%
    {Ablatiestudie over de verschillende onderdelen van een 3D bin packing algoritme.}
\academicyear{\advance\year by -1 \the\year--\advance\year by 1 \the\year}
\examperiod{1}
\degreesought{\IfLanguageName{dutch}{Professionele bachelor in de toegepaste informatica}{Bachelor of applied computer science}}
\partialthesis{false} %% To display 'in partial fulfilment'
%\institution{Internshipcompany BVBA.}

%% Add global exceptions to the hyphenation here
\hyphenation{back-slash}

%% The bibliography (style and settings are  found in hogentthesis.cls)
\addbibresource{bachproef.bib}            %% Bibliography file
\addbibresource{../voorstel/voorstel.bib} %% Bibliography research proposal
\defbibheading{bibempty}{}

%% Prevent empty pages for right-handed chapter starts in twoside mode
\renewcommand{\cleardoublepage}{\clearpage}

\renewcommand{\arraystretch}{1.2}

%% Content starts here.
\begin{document}

%---------- Front matter -------------------------------------------------------

\frontmatter

\hypersetup{pageanchor=false} %% Disable page numbering references
%% Render a Dutch outer title page if the main language is English
\IfLanguageName{english}{%
    %% If necessary, information can be changed here
    \degreesought{Professionele Bachelor toegepaste informatica}%
    \begin{otherlanguage}{dutch}%
       \maketitle%
    \end{otherlanguage}%
}{}

%% Generates title page content
\maketitle
\hypersetup{pageanchor=true}

\input{samenvatting}

%---------- Inhoud, lijst figuren, ... -----------------------------------------

\tableofcontents

% In a list of figures, the complete caption will be included. To prevent this,
% ALWAYS add a short description in the caption!
%
%  \caption[short description]{elaborate description}
%
% If you do, only the short description will be used in the list of figures

\listoffigures

% If you included tables and/or source code listings, uncomment the appropriate
% lines.
%\listoftables
%\listoflistings

% Als je een lijst van afkortingen of termen wil toevoegen, dan hoort die
% hier thuis. Gebruik bijvoorbeeld de ``glossaries'' package.
% https://www.overleaf.com/learn/latex/Glossaries

%---------- Kern ---------------------------------------------------------------

\mainmatter{}

% De eerste hoofdstukken van een bachelorproef zijn meestal een inleiding op
% het onderwerp, literatuurstudie en verantwoording methodologie.
% Aarzel niet om een meer beschrijvende titel aan deze hoofdstukken te geven of
% om bijvoorbeeld de inleiding en/of stand van zaken over meerdere hoofdstukken
% te verspreiden!

\input{inleiding}
\input{standvanzaken}
\input{methodologie}

% Voeg hier je eigen hoofdstukken toe die de ``corpus'' van je bachelorproef
% vormen. De structuur en titels hangen af van je eigen onderzoek. Je kan bv.
% elke fase in je onderzoek in een apart hoofdstuk bespreken.

%\input{...}
%\input{...}
%...

\input{conclusie}

%---------- Bijlagen -----------------------------------------------------------

\appendix

\chapter{Onderzoeksvoorstel}

Het onderwerp van deze bachelorproef is gebaseerd op een onderzoeksvoorstel dat vooraf werd beoordeeld door de promotor. Dat voorstel is opgenomen in deze bijlage.

%% TODO: 
%\section*{Samenvatting}

% Kopieer en plak hier de samenvatting (abstract) van je onderzoeksvoorstel.

% Verwijzing naar het bestand met de inhoud van het onderzoeksvoorstel
%---------- Inleiding ---------------------------------------------------------

\section{Introductie}%
\label{sec:introductie}

Bij het 3D bin packing problem krijg je een set van 3-dimensionalevormen en een aantal containers, beiden met gekende afmetin-gen. De bedoeling is om de vormen te stapelen in het kleinstaantal containers zonder overlappingen zodat zo weinig mo-gelijk lege ruimte overblijft.  (Gonçalves & Resende, 2013)Dit probleem heeft in het dagelijkse leven veel implicatiesdoor de grote hoeveelheid pakketten die worden opgeslagenof getransporteerd. Veel bedrijven hebben dus baat om dit zoefficiënt mogelijk te doen want dit zorgt voor lagere transportkosten en lagere CO2 uitstoot. (Park e.a., 1996)

\begin{itemize}
  \item kaderen thema
  \item de doelgroep
  \item de probleemstelling en (centrale) onderzoeksvraag
  \item de onderzoeksdoelstelling
\end{itemize}


\section{State-of-the-art}%
\label{sec:state-of-the-art}

Een grote uitdaging met het 3D bin packing probleem is degrote hoeveelheid aan mogelijke oplossingen, dit maakt hetonmogelijk om de optimale oplossing te vinden door allecombinaties uit te proberen in een aanvaardbare tijd.   Het3D bin packing problem is NP-compleet. Hierdoor wordt ertypisch gebruik gemaakt van heuristieken die in een zo klein mogelijke tijd een oplossing te vinden die zo nauw mogelijkaanleunt bij de optimale oplossing. (Lodi e.a., 2002)
\subsection{Het packing algoritme van Optioryx}
Optioryx is een bedrijf dat zich bezighoudt met een 3D binpacking algoritme te ontwikkelen. Deze kan worden gebruiktdoor hun klanten om items efficiënter te transporteren en zoook kosten en CO2 uitstoot te verlagen.  De klant geeft deitems en hun afmetingen in samen met de containers waarindeze gestapeld worden en het algoritme geeft een zo efficiëntmogelijke oplossing. Dit kan gebruikt worden voor de itemsin  een  doos  te  steken  of  voor  ze  te  stapelen  op  een  paletof in een container.   Deze toepassingen van het algoritmekomen wel met aanvullende beperkingen. Bij het stapelen vanpakketten in een voertuig moet er rekening worden gehoudenmet het gewicht zodat deze verspreid blijft over het volledigeoppervlak.

\subsection{De onderdelen onderzoeken}
Dit algoritme kan worden opgedeeld in meerdere delen dieelk bijdragen tot het uiteindelijke resultaat. In deze bachelor-proef gaan we deze onderdelen apart onderzoeken en kijkenwat voor impact ze individueel hebben op het resultaat enhoeveel procent van de totale duur van het algoritme het inbeslag neemt. Zo krijgen we een idee welke onderdelen hetbelangrijkst zijn voor de snelheid en efficiëntie van het algo-ritme.  Zo kunnen we ook bepalen waar het algoritme kanworden versneld met verwaarloosbare negatieve impact op hetresultaat. Verder kunnen we nog kijken of er combinaties vanonderdelen een uitzonderlijk verschil in resultaat geven alsdeze samen worden onderzocht.

%---------- Methodologie ------------------------------------------------------
\section{Methodologie}%
\label{sec:methodologie}
Bij het onderzoeken beginnen we met het algoritme af te bake-nen in de verschillende delen zodat elk deel apart kan wordenbekeken.  We zullen dan elk deel vervangen door een sim-pele baseline die dezelfde input neemt en een correcte uitputgeeft. Deze baseline kan bijvoorbeeld de output willekeurigbepalen. Het algoritme wordt dan meerdere keren getest opmeerdere datasets met verschillende karakteristieken. In elkereeks zullen we een onderdeel veranderen door de baselineen het resultaat vergelijken met het effectieve algoritme. Wezullen ook van elke onderdeel de tijd die het erover doet bij-houden.  Door alles te vergelijken kunnen we bepalen waarhet algoritme het best verbeterd wordt en waar we het kun-nen versnellen zonder veel negatieve impact te hebben op hetresultaat.  Of waar het veel efficiënter kan worden gemaaktzonder dat het algoritme veel langer denkt over de oplossing.

%---------- Verwachte resultaten ----------------------------------------------
\section{Verwacht resultaat, conclusie}%
\label{sec:verwachte_resultaten}

Uit dit onderzoek verwachten we te zien dat sommige onderde-len duidelijk zeer belangrijk zijn voor een efficiënte oplossingte vinden. Deze zullen waarschijnlijk ook het meeste tijd inbeslag nemen en het is hier dan ook belangrijk dat ze niet telang denken over een oplossing. Er zullen ook onderdelen zijndie maar een klein procent van de totale tijd in beslag nemen.Deze zullen waarschijnlijk ook het minste invloed hebben enhier is het ook mogelijk dat de baseline in sommige gevallenzelfs beter presteert.

%%---------- Andere bijlagen --------------------------------------------------
% TODO: Voeg hier eventuele andere bijlagen toe. Bv. als je deze BP voor de
% tweede keer indient, een overzicht van de verbeteringen t.o.v. het origineel.
%\input{...}

%%---------- Backmatter, referentielijst ---------------------------------------

\backmatter{}

\setlength\bibitemsep{2pt} %% Add Some space between the bibliograpy entries
\printbibliography[heading=bibintoc]

\end{document}
